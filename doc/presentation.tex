% Variante ohne Aufdeck-Effekte \documentclass[handout]{beamer}
% http://groups.google.de/group/comp.text.tex/browse_thread/thread/6ac8485c7a17252a
\documentclass{beamer}

\usepackage[utf8]{inputenc}
\usepackage[ngerman]{babel}

\usepackage{listings}
\usepackage{color}

\title{Das Carrierpigeon-Projekt}
\author{Hong-Khoan Quach, Marek Kubica, Julius Adorf}
{
\date{\today\\ -  \\ETI-Projekt GP 8 \\ Sommersemester 2009 \\ Technische Universität  München }


\begin{document}


% Entfernen der Navigationsleiste
% Siehe auch http://wiki.ubuntuusers.de/ubuntuusers/LaTeX-Beamer
\beamertemplatenavigationsymbolsempty

% ftp://ftp.tex.ac.uk/tex-archive/macros/latex/.../listings/listings.pdf
\lstset{language=C}
\lstset{numbers=left}
\lstset{
    basicstyle=\small,
    keywordstyle=\color[rgb]{0.2,0.8,0.2}
    }

\frame{\titlepage}

% --------------------------
\section{Idee}

\frame{

    \frametitle{Idee}

}

\section{Architektur}

\frame{

    \frametitle{Architektur}

}

\section{Lösungswege}

\frame{

    \frametitle{Lösungswege}

}

\frame{

    \frametitle{Sicherheit und Stabilität}

    \includegraphics[scale=0.4]{media/bluescr}

}

\frame{

    \frametitle{Herausforderung: Stabilität}

}

\frame{
    \frametitle{Herausforderung: Stabilität}

    \begin{itemize}
        \item Server behält Kontrolle 
        \item Timeouts
        \item Vermeidung globaler Variablen
        \item While-Schleifen sorgfältig überprüfen
    \end{itemize}

}

\frame{

    \frametitle{Herausforderung: sicheres Einschalten}

}

\frame{
    
    \frametitle{Herausforderung: sicheres Einschalten}

}



\section{Fazit}

\frame{

    \frametitle{Fazit}

}

\frame{

    % empty}

}

%
% --------------------------
\section{Toolchain}

\frame{

    \frametitle{Toolchain}

}

\section{Literatur}

\frame{

    \frametitle{Literatur}

}

\end{document}

